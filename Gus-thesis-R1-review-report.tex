\documentclass[11pt]{article}
\date{\vspace{-10ex}}

\usepackage{lineno,hyperref}
\modulolinenumbers[5]

%%%%%%%%%%%%%%%%%%%%%%%%%%%%%%%%%%%%%%%%%%%%%%%%%%%%%%%%%%%%%%%%%%%%%%%%%%
% Added lines by Gustavo Quintana to facilitate internal review process
\usepackage{amsmath, amsfonts, mathtools, hyperref, tikz, pgf, subcaption, mathdots}
\usepackage[a4paper, total={6.5in, 9in}]{geometry}
\usepackage{setspace}
\usetikzlibrary{shapes, arrows, calc, patterns, decorations.pathmorphing, decorations.markings, positioning, external}
\tikzexternalize
\tikzset{external/system call={pdflatex \tikzexternalcheckshellescape -halt-on-error -interaction=batchmode -jobname "\image" "\texsource" && pdftops -eps "\image.pdf"}}
\tikzexternalize[shell escape=-enable-write18]
\usepackage{algorithm, algorithmic}
%\usepackage[noend]{algpseudocode}
\renewcommand{\algorithmicrequire}{\textbf{Input:}}
\renewcommand{\algorithmicensure}{\textbf{Output:}}
%\renewcommand{\algorithmicreturn}{\textbf{Initialize:}}

\makeatletter
\newcommand*\rel@kern[1]{\kern#1\dimexpr\macc@kerna}
\newcommand*\widebar[1]{%
  \begingroup
  \def\mathaccent##1##2{%
    \rel@kern{0.8}%
    \overline{\rel@kern{-0.8}\macc@nucleus\rel@kern{0.2}}%
    \rel@kern{-0.2}%
  }%
  \macc@depth\@ne
  \let\math@bgroup\@empty \let\math@egroup\macc@set@skewchar
  \mathsurround\z@ \frozen@everymath{\mathgroup\macc@group\relax}%
  \macc@set@skewchar\relax
  \let\mathaccentV\macc@nested@a
  \macc@nested@a\relax111{#1}%
  \endgroup
}
\makeatother
%%%%%%%%%%%%%%%%%%%%%%%%%%%%%%%%%%%%%%%%%%%%%%%%%%%%%%%%%%%%%%%%%%%%%%%%%%

%% `Elsevier LaTeX' style
%\bibliographystyle{elsarticle-num}
%%%%%%%%%%%%%%%%%%%%%%%

\begin{document}


\title{Promotors' reports on the first draft of the thesis document.} 

\maketitle


\section*{Ivan Markovsky}

High level comments on the thesis draft:

\begin{enumerate}

\item Don't number the Summary along with the other chapters.

\item EWRLS and TV are not standard acronyms. Don't use them.

\item Reorganize the content so that there are only two levels in the
table of contents (chapter and section). For example in the introduction
either move 2.2.1. Statistical analysis, 2.2.2. Experimental validation,
2.2.3. Affine input estimation to separate sections or don't show them
in the table of contents.
\item In Chapter 3. Preliminaries there is only one section. This is a sign
of bad organization. The section heading should be the chapter heading,
or else add other sections.
\item  The introduction begins with two pages of text that as far as I see
are not a summary of the chapter. This text should form a section on its
own.
\item  Apply in practice what you have learned in the course of Jean-luc
Doumont. In particular, start every chapter with a short summary of what
is in the chapter. Don't do this formally because it is a rule but try
to convey the highlights of the chapter and the links among the sections
in just 1/2 page.

\end{enumerate}

\section*{Rik Pintelon}



\begin{enumerate}

\item 


\item 

\item 

\item 

\item 

\item 

\item 

\item 

\item 

\item 

\item 

\end{enumerate}

\end{document}

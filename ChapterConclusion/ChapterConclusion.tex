%!TEX root = ..\Thesis.tex
\glsresetall

\chapter[Conclusions and future work]{Conclusions and \linebreak future work} \label{chap:concl}

\vfill{}

In this final chapter, we first summarize the contributions made to both parts of the thesis: \textbf{Part \ref{part:Filter}} and \textbf{Part \ref{part:PA}}. Then, we propose
some possible further improvements to extend this work in future research.

\newpage
\section*{Conclusions}

Conclusions - Statistical analysis

 We conducted a statistical analysis of a structured errors-in-variables (EIV) estimation problem with correlation to find the first and second moments of its least-squares solution.
This estimation problem occurs in metrology when we estimate the value of a measurand directly from the sensor transient response.
The data-driven estimation of the physical quantity is formulated as a structured EIV problem with correlation that uses the observed transient response to construct both the regression matrix and the regressor.
The real-time implementation of the method uses a recursive least squares algorithm that is simple and has low computational complexity.
The assessment of the uncertainty is done using the estimate bias and variance.

The conducted statistical analysis produced expressions that predict the estimate bias and variance for given sample size and perturbation level of the observed response.
The Monte Carlo simulation validated the predictions.
We compared the results of solving an unstructured and uncorrelated EIV problem with a structured and correlated EIV problem to understand how the structure and the correlation impacts in the estimation.
We found that the predictions in the structured case are more susceptible to perturbations.
This is due to the two approximations involved, a second-order Taylor series expansion of the estimate, and the substitution of perturbed data on the prediction expressions.
The relative error results indicate that the estimate bias, and variance are predicted using the derived expressions, and the observed data.
The mean squared error of the estimate is close to the results of the maximum likelihood estimate given by the Cram\'er-Rao lower bound.

The bias and variance can be accurately predicted, provided that the Taylor series expansion is valid.
This constraint has to be taken into account to ensure the effectiveness of the method in practical applications.
In the example, it was observed that when the SNR lies outside the validity region, the bias and variance estimation was at most three times larger than the empirical values.

The methodology presented in this paper can be applied to estimate the uncertainty of the solutions to other structured EIV problems.
The bias and variance expressions obtained after the statistical analysis depend on each specific structure.


Conclusions - Experimental validation

In this paper we investigated the statistical properties of a data-driven step input estimation method in a real-life application.
The step input estimation method is a structured and correlated errors-in-variables problem that is solved with recursive least squares.
A statistical analysis was conducted using the ordinary least squares condensed notation. 
This statistical analysis of the input estimate provides expressions that approximate the estimation bias and variance assuming that the measurement noise is Gaussian white noise. 
The variance approximation is useful to assess the uncertainty of the input estimate.
In simulation we observed that the mean squared error of the input estimate is close to the theoretical minimum that uses the Cram\'er-Rao lower bound for biased estimators.
Since the data-driven step input estimation method is not statistically efficient, there is room for improvement. This is a topic for future research.
In the practical experiments, the measurement noise is not white.
The noise variance obtained from the sensor steady state response underestimates the measurement noise variance, that was observed 5 dB larger in the power spectrum due to nonlinearities of the sensor.
Considering this difference in the measurement noise variance, we introduced a conservative bound of the measurement noise variance so that the first and second moments of the input estimate are more accurately predicted.
Using the variance approximation, we can assess the uncertainty of the input estimate with respect to the number of samples processed by the data-driven step input estimation method.
The step input estimation method is useful in practical applications where the whiteness assumption of the measurement noise is not fulfilled. 


Conclusions - Ramp input

An adaptive subspace method was proposed for the estimation of affine input parameters given the measurement of the caused sensor transient response. 
The subspace estimation method is a recursive method that allows online implementation.
This method tracks the input of a system, using exponential forgetting, to process the system response.
The subspace method is model-free and estimates directly the input parameters without identifying a sensor model.
Therefore, it can be applied to the measurement of different physical magnitudes.
In the specific weighing example described in the manuscript, the input is an affine function.
The method is also applicable when the sensor is time-varying.
The subspace method is computationally cheap, simple and suitable for implementation on digital signal processor of low computational power. 

A maximum-likelihood estimator based on local optimization was designed to obtain a comparative reference for the other methods.
The maximum-likelihood method estimates the affine input parameters and also model parameters and the sensor's initial conditions.
This method simulates, in a receding horizon scheme, the response of a sensor model to estimate the input and minimizes the sum of the squares of the residual between the measured and the estimated responses.
The main drawback of the maximum-likelihood method is its computational cost and efficient implementation of the method is left for future work.

A linear time-invariant weighing system is used as an test example for the estimation methods.
The weighing system becomes time-varying when an affine input excites the system.
The estimation methods are compared in a simulation study where the time-varying sensor response is perturbed by measurement noise, that is assumed white of zero mean and known variance.
The subspace method results are also compared to those of an existing digital time-varying filter.
The coefficients of the time-varying filter require offline optimization.
The estimation results obtained with the subspace method converges two times faster and is one order of magnitude smaller than those obtained with the time-varying filter.
The empirical mean squared errors of the subspace method estimation is two orders of magnitude larger than the theoretical minimum given by the Cram\'er-Rao Lower bound.

Future work of this research is the practical implementation of the subspace method for real-time measurements.

\begin{comment}
\section{Conclusions - Ramp input}

An adaptive subspace method was proposed for the estimation of affine input parameters given the measurement of the caused sensor transient response. 
The subspace estimation method is a recursive method that allows online implementation.
This method tracks the input of a system, using exponential forgetting, to process the system response.
The subspace method is model-free and estimates directly the input parameters without identifying a sensor model.
Therefore, it can be applied to the measurement of different physical magnitudes.
In the specific weighing example described in the manuscript, the input is an affine function.
The method is also applicable when the sensor is time-varying.
The subspace method is computationally cheap, simple and suitable for implementation on digital signal processor of low computational power. 

A maximum-likelihood estimator based on local optimization was designed to obtain a comparative reference for the other methods.
The maximum-likelihood method estimates the affine input parameters and also model parameters and the sensor's initial conditions.
This method simulates, in a receding horizon scheme, the response of a sensor model to estimate the input and minimizes the sum of the squares of the residual between the measured and the estimated responses.
The main drawback of the maximum-likelihood method is its computational cost and efficient implementation of the method is left for future work.

A linear time-invariant weighing system is used as an test example for the estimation methods.
The weighing system becomes time-varying when an affine input excites the system.
The estimation methods are compared in a simulation study where the time-varying sensor response is perturbed by measurement noise, that is assumed white of zero mean and known variance.
The subspace method results are also compared to those of an existing digital time-varying filter.
The coefficients of the time-varying filter require offline optimization.
The estimation results obtained with the subspace method converges two times faster and is one order of magnitude smaller than those obtained with the time-varying filter.
The empirical mean squared errors of the subspace method estimation is two orders of magnitude larger than the theoretical minimum given by the Cram\'er-Rao Lower bound.

Future work of this research is the practical implementation of the subspace method for real-time measurements.

\section{Conclusions - Statistical analysis}

 We conducted a statistical analysis of a structured errors-in-variables (EIV) estimation problem with correlation to find the first and second moments of its least-squares solution.
This estimation problem occurs in metrology when we estimate the value of a measurand directly from the sensor transient response.
The data-driven estimation of the physical quantity is formulated as a structured EIV problem with correlation that uses the observed transient response to construct both the regression matrix and the regressor.
The real-time implementation of the method uses a recursive least squares algorithm that is simple and has low computational complexity.
The assessment of the uncertainty is done using the estimate bias and variance.

The conducted statistical analysis produced expressions that predict the estimate bias and variance for given sample size and perturbation level of the observed response.
The Monte Carlo simulation validated the predictions.
We compared the results of solving an unstructured and uncorrelated EIV problem with a structured and correlated EIV problem to understand how the structure and the correlation impacts in the estimation.
We found that the predictions in the structured case are more susceptible to perturbations.
This is due to the two approximations involved, a second-order Taylor series expansion of the estimate, and the substitution of perturbed data on the prediction expressions.
The relative error results indicate that the estimate bias, and variance are predicted using the derived expressions, and the observed data.
The mean squared error of the estimate is close to the results of the maximum likelihood estimate given by the Cram\'er-Rao lower bound.

The bias and variance can be accurately predicted, provided that the Taylor series expansion is valid.
This constraint has to be taken into account to ensure the effectiveness of the method in practical applications.
In the example, it was observed that when the SNR lies outside the validity region, the bias and variance estimation was at most three times larger than the empirical values.

The methodology presented in this paper can be applied to estimate the uncertainty of the solutions to other structured EIV problems.
The bias and variance expressions obtained after the statistical analysis depend on each specific structure.

\section{Conclusions - Experimental validation}

In this paper we investigated the statistical properties of a data-driven step input estimation method in a real-life application.
The step input estimation method is a structured and correlated errors-in-variables problem that is solved with recursive least squares.
A statistical analysis was conducted using the ordinary least squares condensed notation. 
This statistical analysis of the input estimate provides expressions that approximate the estimation bias and variance assuming that the measurement noise is Gaussian white noise. 
The variance approximation is useful to assess the uncertainty of the input estimate.
In simulation we observed that the mean squared error of the input estimate is close to the theoretical minimum that uses the Cram\'er-Rao lower bound for biased estimators.
Since the data-driven step input estimation method is not statistically efficient, there is room for improvement. This is a topic for future research.
In the practical experiments, the measurement noise is not white.
The noise variance obtained from the sensor steady state response underestimates the measurement noise variance, that was observed 5 dB larger in the power spectrum due to nonlinearities of the sensor.
Considering this difference in the measurement noise variance, we introduced a conservative bound of the measurement noise variance so that the first and second moments of the input estimate are more accurately predicted.
Using the variance approximation, we can assess the uncertainty of the input estimate with respect to the number of samples processed by the data-driven step input estimation method.
The step input estimation method is useful in practical applications where the whiteness assumption of the measurement noise is not fulfilled. 

\section*{Future work}
\end{comment}

text

\newpage

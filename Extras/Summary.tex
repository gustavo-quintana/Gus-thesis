%!TEX root = ..\Thesis.tex
\chapter{Summary} \label{chap:Summary}
\vspace{-1.5cm}
Measurements are dynamical processes where the input is estimated using the sensor response. 
The users cannot wait for the sensor to reach a steady regime when the sensor response closely approximates the sensor static gain times the input. 
To obtain a fast input estimation, one approach is filtering the sensor transient response with another dynamical system, aiming to compensate the estimation time.
The compensator is commonly built after a sensor model to perform deconvolution.  
Another approach is estimating the input with data-driven methods that are independent from the sensor model.
Current digital signal processors (DSP) allow the implementation of both model-based and data-driven estimation methods.

There exists a data-driven method that directly estimates the step input level from the sensor step response. 
The formulation of this method is a Hankel-structure errors-in-variables (EIV) problem, where the regression matrix and the regressor are correlated.
The structured EIV problem is solved with recursive least-squares for real-time implementation.
The independence from the sensor model of the estimation method broadens its range of application.

The uncertainty of the step input estimation method is unknown and is not straightforwardly evident.
To validate the method for metrology applications, we studied the stochastic properties of the step input estimation.
A statistical analysis yields the estimation bias and variance, and thus, the estimation uncertainty.

Both with simulations and experiments, the sample mean squared error (MSE) of the step input estimation method is compared to its analytical Cram\'er-Rao bound.
The Cram\'er-Rao bound determines the minimum theoretical variance of the estimation.
It was found that the data-driven step input estimation is biased but has small variance, and the estimation MSE is less than one order of magnitude larger than the Cram\'er-Rao bound.
Moreover, in practical experiments, the step input estimation method is robust when the measurement noise is not Gaussian and white.

\begin{comment}
\section{Summary - Ramp input}

A measurement is a dynamical process that aims to estimate the true value of a measurand.
The measurand is the input that excites a sensor, and, as a consequence, the sensor output is a transient response. 
The main approach to estimate the input is applying the sensor transient response to another dynamical system. 
This dynamical system is designed by deconvolution to invert the sensor dynamics and compensate the sensor response. 
Digital signal processors enable an alternative approach to estimate the unknown input.
There exists a data-driven subspace-based signal processing method that estimates a measurand, assuming it is constant during the measurement.
To estimate the parameters of a measurand that varies at a constant rate, 
we extended the data-driven input estimation method to make it adaptive to the affine input.
In this paper, we describe the proposed subspace signal processing method for the measurement of an affine measurand and compare its performance to a maximum-likelihood input estimation method and to an existing time-varying compensation filter.
The subspace method is recursive and allows real-time implementations since it directly estimates the input without identifying a sensor model.
The maximum-likelihood method is model-based and requires very high computational effort.
In this form, the maximum-likelihood method cannot be implemented in real-time, however, we used it as a reference to evaluate the subspace method and the time-varying compensation filter results.
The effectiveness of the subspace method is validated in a simulation study with a time-varying sensor.
The results show that the subspace method estimation has relative errors that are one order of magnitude smaller and converges two times faster than the compensation filter.


\section{Summary - Statistical analysis}

A structured errors-in-variables (EIV) problem arising in metrology is studied.
The observations of a sensor response are subject to perturbation. 
The input estimation from the transient response leads to a structured EIV problem.
Total least squares (TLS) is a typical estimation method to solve EIV problems.
The TLS estimator of an EIV problem is consistent, and can be computed efficiently when the perturbations have zero mean, and are independently and identically distributed (i.i.d).
If the perturbation is additionally Gaussian, the TLS solution coincides with maximum-likelihood (ML).
However, the computational complexity of structured TLS and total ML prevents their real-time implementation.
The least-squares (LS) estimator offers a suboptimal but simple recursive solution to structured EIV problems with correlation, but the statistical properties of the LS estimator are unknown.
To know the LS estimate uncertainty in EIV problems, either structured or not, to provide confidence bounds for the estimation uncertainty, and to find the difference from the optimal solutions, the bias and variance of the LS estimates should be quantified.
Expressions to predict the bias and variance of LS estimators applied to unstructured and structured EIV problems are derived.
The predicted bias and variance quantify the statistical properties of the LS estimate and give an approximation of the uncertainty and the mean squared error for comparison to the Cram\'er-Rao lower bound of the structured EIV problem.

\section{Summary - Experimental validation}

Simultaneous fast and accurate measurement is still a challenging and active problem in metrology.
A sensor is a dynamic system that produces a transient response.
For fast measurements, the unknown input needs to be estimated using the sensor transient response.
When a model of the sensor exists, standard compensation filter methods can be used to estimate the input.
If a model is not available, either an adaptive filter is used or a sensor model is identified before the input estimation. 
Recently, a signal processing method was proposed to avoid the identification stage and estimate directly the value of a step input from the sensor response.
This data-driven step input estimation method requires only the order of the sensor dynamics and the sensor static gain.
To validate the data-driven step input estimation method, in this paper, the uncertainty of the input estimate is studied and illustrated on simulation and real-life weighing measurements.
It was found that the predicted mean-squared error of the input estimate is close to an approximate Cram\'er-Rao lower bound for biased estimators.
\end{comment}


%\newpage
